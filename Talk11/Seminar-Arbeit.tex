\chapter{Botnet Economics and Business Models}
\markboth{Botnet Economics and Business Models}{}
\chaptauthors{Famos Tobias, Mannhart Thomas ,Tham David, Waltert Gian }

\Kurzfassung{%
This is the abstract.
It fits pretty much on one page and is definitely not longer.}

\newpage

\minitoc %table of contents

\newpage

\section{Introduction}

\subsection{Motiviation}
Online Crime has shifted in the last 20 years. Where earlier attacks and malicious actions were performed by individual hackers, trying to break things mostly for fun there are now organized groups build like companies that act as malicious actors in the cyber space. (Copy some of economics of online crime introduction)
Cybercrime is a field of high revenue. It is estimate that the global revenue from cybercrime is at least 1.5 Trillion Us Dollars [5]. In context, that would be roughly equivalanet to the GDP of Australia in the year 2018. [6]
The Damage done by cybercrime is estimated to have risen about 12 % from 2017 to 2018 [1] and in total numbers estimated to be 600 Billion US Dollars [4]. 

Some of the currently big threats in the cyber security landscape are coupled to Botnets, a network of computers controlled by attackers (Putman et al Business). Botnets can be used for various malicious intents and actions such as Distributed Denial of Services Attacks, Spam Generation, Information Theft and data exfiltration. 
There is an increasing number of devices connected to the internet every day. Furthermore, there are predictions that there will be 75 Billion IoT devices in the world by the year of 2025. 
And many of them are poorly secured, if at all. This leads to a serious incentive for Botnet providers to target not only personal computers and smartphones, but also IoT devices such as security cameras, washing machines or doorbells. Devastating examples of the abuse of vulnerabilities in IoT devices, such as default or hardcoded credential can be found in mass. Huge botnets like Linux.Aidra, Mirai or Bashlite infect tens of thousands of devices and use them for malicious purposes.   
The high number of Devices connected to the internet, paired with the trend for more bandwith and manifacturing devices with more CPU power and better Network cards to output higher bandwiths gives malicious actors more possiblities to abuse foreign infrastructure and devices. 
Aditionally there is a lack of incentives for Device Owners to secure their devices, since there is no direct, preceptible impact from an insecure device onto them, or so they think. 

\subsection{Malware in numbers}
As the revenue from cybercrime is growing, the development of malware is too. Researchers estimate, that there are 300000 to 1Million new Viruses developed every day [4] Currently, the fastest growing malware thread is ransomware [4]. The FBI estimates that there are more than 4000 ransomware attacks preformed every day, with 209 Million US Dollars Ransom payed in the first quarter of 2016. This is a rapid growth compared to the 24 Million Dollars in ransom payment in the whole year 2015. 
The deployment of Malware onto personal Computers is also changing and growing. The Symantec Threat report states that the primary delivery of Malware has changed in recent years, from malicious URLs to Malicious attachments. From the Malicious attachments, 48 % are Office Files with Macros used to download Malware. 
A interesting rise in the cyber crime landscape is the mobile ransomware. These malwares target mobile devices, and lock them down. [7]
Furthermore the threads concerning IoT devices are equally rising. Symantec has seen an increase in internet of things attacks in the year 2017 and the attack numbers stabalizing in the year 2018 [7]. The most attacked devices in their IoT honeypot were routers and security cameras. Togeter they accounted for 90% of the IoT attacks, with routers having 75% and security cameras at 15% [7]. In terms of protocols was Telnet the mostly used one with 90 % of attempted attacks being done using it in 2018. This is up from50% in the year 2017 [7]. 
In terms of threads, the top three accounted for over 75% of the attacks on the IoT honeypots. Namely they were Linux.Lightaidra, Linux.Kaiten and Linux.Mirai. Thus the Mirai Botnet still is an active thread with 15% of attacks originating in it. [7] 
Overall one can state, that Phishing is still the most popular and easiest way to commit a cybercrime. Due to the low costs of Phishing and the perceived low risk of getting caught phishing remains an attractive attack vector.[4]

 
\subsubsection{Oranisation / Outline}
In the following Paper, we want to show the importance of further research about Botnets and the importance of Botnets itself. By giving a brief overview about the topic of Botnets and their possible usage we want to add to the awareness of the reader about the dangers of botnets and the importance of the influence of botnets on the economy. This is underlined by applying the stated analysis and facts onto a botnet in the case study. 


		
\section{Background}
	\subsection{Botnet}
		\subsubsection{Definition}
			A botnet is a network of infected end-hosts called bots (also referred to as zombies or drones) under the control of a botmaster. Botnets recruit vulnerable machines using basic methods of malware distribution. The botmaster is the manager of the botnet and controls his bots by using command and control (C\&C) channels. These channels are used by the botmaster to distribute his commands to his bots. These channels can use multiple communication mechanisms like P2P or Internet Relay Chat (IRC). A vast majority are using IRC [1].
			
		\subsubsection{Malware}
Malware stands for malicious software and describes any software that is designed to damage or exploit the infected machine. In our case, the malware is the code installed on the vulnerable machine, that turns it into a bot and allows the botmaster to take control.
Malware spreads using multiple techniques and one specific malware can use more than one of these.

A virus is a malicious piece of code that hides inside a host program. It replicates itself when the host program is executed and inserts its own code into other programs.

Worms are standalone programs that replicate and spread themselves when introduced to a computer network.

The name Trojan horse or Trojan describes any malware, that hide their true and malicious intent. This can be a seemingly useful program, an unsuspicious e-mail attachment or an interesting advertisement.


		\subsubsection{Revenue Models}
In today's botnet economy, the attacker using a botnet is rarely the creator and botmaster of the used botnet. Botnets are traded and rented, to generate revenue without performing attacks [2]. There already exist services for development, distribution and hosting of botnets. In many cases, this even includes customer support. Bottazi and Me (2014) call this model Cybercrime-as-a-Service (CaaS) [3].

With Distributed Denial-of-Service (DDoS) attacks revenue can be generated by executing attacks on competitors in the market and getting a part of their market share or executing attacks on behalf of a competitor in exchange for payment.
Another possibility is cyber extortion, by threatening a DDoS attack if a certain amount of money is not paid to the attacker.

Theft and fraud: Botnets are responsible for the majority of spam distributed through the internet (i.e. phishing mail). They are used to host fake websites (phishing websites) to steal personal and valuable information and they enable click fraud [4]. Click fraud is the exploitation of pay-per-click online advertising by imitating a real user and clicking on the advertiser's link. The advertiser has to pay the publisher and the advertising network, therefore creating a potential conflict of interest.

		\subsubsection{Command and Control}
Command and control channels (C\&Cs) are used to communicate with the bots in a botnet. Those channels are based on basic internet communication protocols. The majority of known botnets are using C\&Cs based on the IRC (Internet Relay Chat) protocol. The botmaster can send and receive massages to and from his bots through a centralized command and control mechanism. The communication is in real-time and is highly successful. There are known botnets using the HTTP protocol for their C\&C. This approach is still centralized but the botmaster cannot directly interact with his bots. The bots have to contact the C\&C server to get their commands [5].

		\subsubsection{Architecture}
The architecture refers to the model of communication between the botmaster and his bots. We divide these models into three categories: centralized, decentralized, hybrid and unstructured.

In the centralized model, there is one (or few) central point(s) responsible for the exchange of commands and data between the botmaster and his bots. This central point is the C\&C server and runs on a machine with a high bandwidth connection. This server runs a communication service (mostly IRC). This model provides a small latency which makes it easy to control the botnet. A disadvantage of this model is the high vulnerability of the botnet communication. because this central point is responsible for the whole C\&C communication. If this central point gets discovered and eliminated, it renders the whole botnet useless. This threat can be reduced by using multiple redundant servers controlling the same botnet [6].

The decentralized model is a way to eliminate the vulnerabilities of the centralized approach. By using a P2P pattern for C\&C, the system no longer depends on a few selected servers. The bots are interconnected and function as host and as server simultaneously. Each bot knows a fraction of the botnet to send and receive the commands of the botmaster. So even if some bots are detected and taken down, the rest of the botnet continues to receive commands. Zeidanloo and Manaf (2009) stated, that the use of P2P based communication in botnets will be used dramatically in the near future, because botnets using this form of communication are much more challenging to detect and destroy [6].

Wang et al. (2010) proposed a hybrid model that uses the best of both worlds. Such a botnet would contain two types of bots. The Servant Bots, which serve as clients and hosts have static and routable IP addresses. These Servant Bots are accessible from the whole internet, which means they are not behind a firewall that restricts incoming traffic. The Client Bots do not accept incoming connections and have a peer list containing only Servant Bots. The Client bots connect regularly to the Servant Bots in their peer list and forward any new commands to the whole list [7].

An unstructured communication model is based on the principle, that if a bot receives a command from the botmaster, it randomly scans the internet for other bots to propagate the massage. In this case the rest of the botnet will not be affected if a bot gets discovered. The drawbacks are an extremely high latency, the noticeable scanning and that it cannot be guaranteed, that every bot in the botnet receives the command [8].

		\subsubsection{Underground Markets}
As Thomas et al. (2006) state in their article, that there are entire IRC networks dedicated to cybercrime. Those IRC servers are not hidden, but easy to find and easily accessible. The participants in these underground networks often use encryption to hide their identity.
Most of these networks have channels for helping new members and reporting fraud. Members known to have committed fraud are recorded and shared on the network as a form of self-policing.
There are also other such underground networks using HTTP, Instant Massaging, Peer-to-Peer (P2P) and other forms of communication [*].

		\subsubsection{Distributed Denial-of-Service Attacks (DDoS)}
A Denial-of-Service (DoS) attack attempts to prevent the legitimate use of a service by either overwhelming the service with a huge amount of traffic or exploiting an application or protocol by sending malformed packages and causing the service to freeze or reboot. A Denial-of-Service (DoS) attack becomes a Distributed Denial-of-Service (DDoS) attack, if the source of the attack are multiple distributed entities or rather bots in a botnet [9].

An increasingly popular form of DDoS attacks are DDoS amplification attacks, like the memcached DDoS attack on GitHub on February 28, 2018. The attacker abused memcached instances to amplify the attack by up to 51'000 times the originally sent data. The memcached's response has been targeted to addresses used by GitHub.com using IP address spoofing. This amplification resulted in a peak of 1.35Tbps via 126.9 million packets per second sent to GitHub's services [10].

	\subsection{DDoS Defense Systems}
		\subsubsection{Technical Solutions}
\textbf{\textit{a. Aggregate-based congestion control (ACC)}}\\
Mahajan et al. (2002) introduced two aggregate-based congestion control (ACC) mechanisms. The job of the first, local ACC is to identify the aggregates responsible for the congestion of a service and to throttle the throughput of these.

The identification of the offending aggregates is very difficult. The overload may be chronic due to an under-engineered network or unavoidable because the load shifted due to routing changes. The traffic might cluster in multiple dimensions (e.g. a particular server or network link) and the attacker may change their target to avoid detection.

The next challenge is to determine how much an aggregate should get throttled. The goal is to keep the service running during an attack, that implies they cannot block those aggregates completely. The rate limit is chosen so that the remaining traffic can at least maintain some level of service.

The second ACC mechanism is the pushback mechanism. The congested router pushes the aggregate throttling upstream to the adjacent routers sending him a significant fraction of the congesting traffic. This pushback propagates upstream to save bandwidth by dropping packets early and to focus on the routes or routers responsible for the congestion and still letting legitimate traffic through.

The pushback will not be effective against a DDoS attack, if the traffic is distributed evenly across the inbound connections. This can be possible, if the attacker uses an amplification (or reflector) attack with sufficiently distributed reflectors. Pushback can also overcompensate by throttling aggregates upstream that would not have been a problem downstream. Or if the pushback algorithm is not able to differentiate between malicious and legitimate traffic coming from the same edge network, that does not support the pushback mechanism. This fact could even be abused by blocking legitimate incoming traffic from a particular source by launching the attack from a host that is close to this source [11].

\textbf{\textit{b. Max-Min Fair Server-Centric Router Throttles}}\\
Yau et al. (2005) proposed and tested a router throttle mechanism that aims to increase the precision in blocking malicious and protecting legitimate traffic. This is a server-initiated proactive approach, where the server protects itself by installing a router throttle at an upstream router. The throttles depend on the current demand and have to be negotiated dynamically between server and network.

If a server load crosses the load limits, the server installs the throttle at some of its upstream routers. There is a defined upper and lower limit and if reached, the throttle rate is reduced further or increased again.

The fair throttle algorithm tries to minimize the throttling on well behaving routers, by not just decreasing the allowed traffic from every router, but by multicasting a uniform leaky bucket rate (i.e. the throttle rate) to all targeted routers. This way, busier routers drop much more packets then calm ones, which may not have to drop any [12].

\textbf{\textit{c. Probabilistic Packet Marking}}\\
Park and Lee (2001) discussed the effectiveness of IP traceback mechanisms based on probabilistic packet marking (PPM). The aim of probabilistic packet marking is to reduce the overhead produced by deterministic packet marking (DPM).In DPM, each router inscribes its local path information onto the passing packet. Those markings allow the destination node to trace back the traversed path of a packet but increases the packet header size linearly with every hop.

In PPM, the packet header provides a constant space for traceback information and after every hop, the router overwrites this information with a probability p < 1. This allows the destination node to trace back the whole path of the incoming packets, if the attack volume is high enough but also leaves the possibility open, that some packets still contain the original markings of the attacker, who can try to confuse the traceback.

In the case of DDoS attacks with a high number of attacking sources, this technique gets rendered more and more useless. The uncertainty grows more and more while the probability to find the sources becomes smaller and smaller [13].

\textbf{\textit{d. Hash-Based IP Traceback}}\\
Snoeren et al. (2001) present a hash-based IP traceback technique, that allows to trace back the origin of a single IP packet. They developed the Source Path Isolation Engine (SPIE). The engine enables to trace back a packet with a copy of the packet, its destination and an approximate reception time. The main goals of this technique is to reduce memory requirements and to not increase the vulnerability to eavesdropping of a network.

To achieve those goals, SPIE produces a 32-bit packet digest using a hash function. As hash input, it uses the invariant portion of the IP header and the first 8 bytes of the payload. Because it is not possible to store the digest of every packet forwarded on the router, SPIE uses a Bloom filter. This filter computes k distinct digests using k distinct hash functions. The hash function have to be uniform. The n-bit results get indexed to a bit array of the size 2n. If the copy of the file to trace back gets indexed to only 1-bits, with high possibility, the packet got forwarded by this router [14].

\textbf{\textit{e. Packet Filtering using Traffic Quota}}\\
The DDoS Defense System proposed by Xu and Lee (2003) assumes that the firewall is the bottleneck during a DDoS attack. The Server and Client do not need to be changed and the firewall issues the filtering operation. The goal is to produce a line of defense in the local network with a set of routers, typically belonging to a local ISP. Those so-called perimeter routers should distinguish between DDoS and legitimate traffic. They distinguish two types of DDoS attacks, the first where the attacker uses IP spoofing and the second where the legitimate IP addresses are used.

In the first case, if the first packet of a client arrives, the client gets redirected to a pseudo-IP address and port number pair. This new destination IP address contains a Massage Authentication Code (MAC) that's encrypted with a symmetric key shared between the firewall and the perimeter routers. If the senders IP address is spoofed, the HTTP redirect message will never reach him.
To avoid the collecting of valid MAC's from legitimate clients, the Key changes over time and gets a small timestamp or version number.

in the second case, where the attacker uses the real IP addresses, they cannot be distinguished from legitimate customers. The firewall has the job to allocate the available bandwidth fairly between all clients. Using a Deficit Round Robin algorithm, every client gets a traffic quota and all excess packets are dropped. The firewall tracks the amount of dropped packets and blacklists clients overreaching a particular threshold. To defend against a continuous attack that stays in the traffic quota, a "no loitering" law is enforced. If the total amount of packets sent by a client is bigger than a given quota, its traffic quota gets reduced to a fraction.

The main problems are, that a IP spoofed DDoS attack can make it hard for legitimate clients to get their first package through to get redirected and if all attackers use their fair share using genuine IP address, the service can suffer a response time too big for real customers to deal with [15].

\textbf{\textit{f. Hop-count Filering}}\\
Jin et al. (2003) propose a hop-count-based approach to filter out spoofed IP packets. They use a victim-based approach that does not need the support of the ISP's. The only information needed is contained in the package header, assuming that in most cases of IP spoofing the hop-count of the package does not correspond to the expected hop-count of the IP address. The hop-count can be inferred from the TTL field of the incoming packets. To avoid dropping packets of legitimate clients because the hop-count information is not correct, the filtering only takes place if an attack gets detected [16]. 

		\subsubsection{Economic DDoS Defense (Honeypots)}
Defending against DDoS attacks with technical solutions results in an arms race between attacker and defender. It has become more common to not only look at technical solutions against direct DDoS attacks, but to target the economic aspects of botnets.

Ford and Gordon (2006) describe their Multihost Adware Revenue Killer (MARK), to attack the revenue streams of botnets used for unwanted advertisement and software installs on the victim's machine (Adware and Spyware). This use of botnets is thought to be the most profitable for botmasters. MARK consists of Virtual Machines (VMs) which actively try to get infected by Ad- and Spyware, so called honeypots. Because no one sees the ads shown on those VMs and there is no personal information to find, the botmaster's revenue increases first, but the advertisement or spyware distributer loses money. This destabilization can lead to a massive decrease in payment (i.e. per click or per download), and therefore a massive decrease in revenue for the botmaster when the VMs get cleared and the honeypots are no longer generating revenue [17].

Li et al. (2009) propose a similar strategy, with the goal to make it more troublesome for botmasters to maintain their botnets. They focus more on DDoS attacks and propose to use honeypots to introduce uncertainties to the optimizing problem of botmasters and clients. The honeypots do not participate in DDoS attacks, for this reason the required number of bots for a successful attack increases and the value of a single bot decreases by consistent maintenance costs. According to their model the profits of attackers and botmasters can decrease dramatically [2].


	\subsection{Business Model}
		\subsubsection{Stakeholders}
			Text
		\subsubsection{Life cycle}
			Text
		\subsubsection{Cost Structure}
			Text
		\subsubsection{Revenue Streams}
			Text
		\subsubsection{Channels}
			Text
		\subsubsection{Customers}
			Text
	\subsection{Economic Impact}
		\subsubsection{Metrics}
			Text
		\subsubsection{Comparison of Types}
			Text
	\subsection{Solutions}
		\subsubsection{Incentives to defend}
			Text
		\subsubsection{Technical Solutions}
			Text
		\subsubsection{Economic Solutions}
		
\section{Case Study 1}
	\subsection{Description}
	Text
	\subsection{Business Model}
	Text
	\subsection{Economic Impact}
	Text

\section{Case Study 2}
	\subsection{Description}
	Text
	\subsection{Business Model}
	Text
	\subsection{Economic Impact}
	Text

\section{Evaluation and Discussion}
Text

\section{Summary}
Text
